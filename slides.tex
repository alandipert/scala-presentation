\usecolortheme{lily}
% we need these because of source-highlight output
\DefineNamedColor{named}{Blue}          {cmyk}{1,1,0,0}
\DefineNamedColor{named}{BrickRed}      {cmyk}{0,0.89,0.94,0.28}
\DefineNamedColor{named}{Brown}         {cmyk}{0,0.81,1,0.60}
\DefineNamedColor{named}{ForestGreen}   {cmyk}{0.91,0,0.88,0.12}
\DefineNamedColor{named}{Purple}        {cmyk}{0.45,0.86,0,0}
\DefineNamedColor{named}{Red}           {cmyk}{0,1,1,0}
\DefineNamedColor{named}{Black}         {cmyk}{0,0,0,1}
\DefineNamedColor{named}{RoyalBlue}     {cmyk}{1,0.50,0,0}
\DefineNamedColor{named}{TealBlue}      {cmyk}{0.86,0,0.34,0.02}
\DefineNamedColor{named}{CarnationPink} {cmyk}{0,0.63,0,0}

\title{Scala, a Scalable Language} 
\author{Alan Dipert} 
\date{September 10th, 2009} 

\begin{document} 
\maketitle 

\begin{frame} 
\frametitle{What Scala Is}
\begin{columns}[c]
  \column{0.5in}
    \includegraphics[width=1.0in]{graphics/scala_logo.png} 
  \column{2.5in}
    \begin{itemize}
      \item<1-> Multi-paradigm: purely object oriented, but with functional capabilities
      \item<2-> Statically typed
      \item<3-> Seamless Java interoperability
      \item<4-> Open source (BSD License)
      \item<5-> Latest of several JVM languages from creator Martin Odersky and team at EPFL
      \item<6-> Cool: \tt\small{def fact(x:BigInt):BigInt = if (x==0) 1 else x * fact(x-1)}
    \end{itemize}
\end{columns}
\end{frame} 

\begin{frame} 
\frametitle{Object Oriented Programming in Scala}
\begin{columns}[c]
  \column{1.0in}
    \includegraphics[width=1.8in]{graphics/oop_dot.pdf} 
  \column{2.0in}
    \begin{itemize}
      \item<1-> Classes and objects similar to Java...
      \item<2-> ...except "everything is an object." - no primitive types like Java
      \item<3-> Automatic getters/setters (like :attr\_accessor in Ruby)
      \item<4-> "Singleton objects" instead of static methods in classes
      \item<5-> "Traits," behaviors mixed in with classes and objects, simplify multiple inheritance
    \end{itemize}
\end{columns}
\end{frame} 

\begin{frame}
\frametitle{Class Definition in Scala}
\tt\small{
\input{code/person1.scala.tex}
}
\end{frame} 

\begin{frame}
\frametitle{Inheritance in Scala}
\tt\tiny{
\input{code/person2.scala.tex}
}
\end{frame} 

\begin{frame}
\frametitle{Traits in Scala}
\tt\tiny{
\input{code/person3.scala.tex}
}
\end{frame} 

\begin{frame} 
\frametitle{Object Oriented Programming Conclusions}
\begin{columns}[c]
  \column{1.0in}
    \includegraphics[width=1.5in]{graphics/ferris.jpg} 
  \column{2.0in}
    \begin{itemize}
      \item<1-> Scala's object system gives you a lot of control
      \item<2-> It feels much more like Ruby or Python than Java or C++
      \item<3-> It's a carefully designed fusion of useful features and Java compatibility
      \item<4-> Like a lot of Scala, you can start by doing things the Java way
    \end{itemize}
\end{columns}
\end{frame} 

\begin{frame} 
\frametitle{Functional Programming Introduction}
\begin{columns}[c]
  \column{0.5in}
    \includegraphics[width=1.0in]{graphics/hat.png} 
  \column{2.5in}
    \begin{itemize}
      \item<1-> FP is the paradigm of avoiding state and mutability in code, "functions and data the same"
      \item<2-> Around in practical form since Lisp (1958), popular lately because of renewed interest in concurrency
      \item<3-> Side-effect free functions and immutable data structures have definite advantages in many domains
      \item<4-> On the JVM, FP alternatives are Clojure, CAL
      \item<5-> In Java terms, anonymous inner classes on crack
    \end{itemize}
\end{columns}
\end{frame} 

\begin{frame} 
\frametitle{Functional Programming in Scala}
\begin{itemize}
  \item<1-> Scala has first-class functions
  \item<2-> Many Collection classes provide convenience higher order functions: \tt\small{reduceLeft, foldLeft, filter, forall}
  \item<3-> Lots of functional stand-bys: \tt\small{head, tail}
  \item<4-> Functional techniques help you avoid off-by-one errors, temporary variables, and explicit loops
  \item<5-> Chained higher-order functions let you write common imperative for/if patterns concisely
\end{itemize}
\end{frame} 

\begin{frame} 
\frametitle{Functional Programming in Scala Examples}
\begin{itemize}
  \item<1-> Sum a list of Integers:
  \item<1-> \tt\small\textbf{List(1,5,3,2).reduceLeft((x,y) => x+y) (Int = 11)}
  \item<2-> More concisely:
  \item<2-> \tt\small\textbf{List(1,5,3,2).reduceLeft(\_+\_) (Int = 11)}
  \item<3-> Prune odd numbers:
  \item<3-> \tt\small\textbf{List(1,2,3,4).filter(x => x \% 2 == 0) (List[Int] = List(2,4))}
  \item<4-> Prune odd numbers, get sum:
  \item<4-> \tt\small\textbf{List(1,2,3,4).filter(\_ \% 2 == 0).reduceLeft(\_+\_) (Int = 6)}
  \item<5-> Or, in Scala 2.8:
  \item<5-> \tt\small\textbf{List(1,2,3,4).filter(\_ \% 2 == 0) sum (Int = 6)}
\end{itemize}
\end{frame} 

\begin{frame} 
\frametitle{Putting it All Together: a Simple Application}
\begin{itemize}
  \item<1-> A program that adds a list of numbers taken as a command line argument, in three languages 
  \item<2-> Imperative (C), imperative/OO (Java), and OO/functional (Scala)
  \item<3-> \textbf{Warning: always use the right tool for the job!}
  \item<3-> This is just a demonstration.
  \item<4-> Our program is supposed to take one argument, a comma-delimited list of integers to sum.
  \item<4-> Usage: \tt{./summer 1,2,3} should print 6
\end{itemize}
\end{frame} 

\begin{frame} 
\frametitle{Old Reliable: C}
\tt\tiny{
\input{code/summer.c.tex}
}
\end{frame} 

\begin{frame} 
\frametitle{The Clinton Years: Java}
\tt\tiny{
\input{code/Summer.java.tex}
}
\end{frame} 

\begin{frame} 
\frametitle{Today: Scala}
\tt\tiny{
\input{code/Summer.scala.tex}
}
\end{frame} 

\begin{frame} 
\frametitle{Bonus!: Ruby}
\tt\tiny{
\input{code/Summer.rb.tex}
}
\end{frame} 

\begin{frame} 
\frametitle{Thank You!}
\begin{itemize}
  \item<1-> \href{http://www.scala-lang.org/}{The Scala Homepage [http://www.scala-lang.org/]}
  \item<1-> \href{http://www.artima.com/shop/programming\_in\_scala}{Programming in Scala, the definitive book [http://www.artima.com/shop/programming\_in\_scala]}
  \item<1-> \href{http://github.com/alandipert/scala-presentation/tree/master}{git repository for this presentation [http://github.com/alandipert/scala-presentation/tree/master]}
\end{itemize}
\begin{itemize}
  \item<1-> \href{mailto:alan.dipert@gmail.com}{alan.dipert@gmail.com}
  \item<1-> \href{http://alan.dipert.org/}{http://alan.dipert.org/}
\end{itemize}
\end{frame} 

\end{document} 
